\documentclass{article}
\usepackage[spanish]{babel}
\usepackage[utf8]{inputenc}
\usepackage{amsmath}   % Para ecuaciones avanzadas
\usepackage{amssymb}   % Símbolos matemáticos

\begin{document}

\section{Ecuaciones en LaTeX}

\subsection{Ecuación en línea}
La famosa relación entre energía y masa se expresa como $E = mc^2$,
la cual es una de las ecuaciones más conocidas de la física.

\subsection{Ecuación centrada sin numerar}

\[
E = mc^2
\]

\subsection{Ecuación numerada}

\begin{equation}
E = mc^2
\label{eq:energia}
\end{equation}

Como se muestra en la ecuación~\ref{eq:energia}, la energía es proporcional
a la masa y al cuadrado de la velocidad de la luz.

\subsection{Fracciones, raíces y potencias}

\begin{equation}
E = \frac{1}{2}mv^2
\label{eq:cinetica}
\end{equation}

La ecuación~\ref{eq:cinetica} representa la energía cinética de un cuerpo.

\subsection{Ecuaciones de varias líneas}

\begin{align}
F &= ma \label{eq:newton} \\
E &= mc^2
\end{align}

La ecuación~\ref{eq:newton} corresponde a la segunda ley de Newton.

\subsection{1. Integral definida}

La integral definida se expresa como:

\begin{equation}
\int_{a}^{b} f(x)\,dx = F(b) - F(a)
\end{equation}

---

\subsection{ Serie de Fourier}

Una función periódica puede representarse como:

\begin{equation}
f(x) = a_0 + \sum_{n=1}^{\infty} 
\left( a_n \cos\left(\frac{2\pi n x}{L}\right) 
+ b_n \sin\left(\frac{2\pi n x}{L}\right) \right)
\end{equation}

---

\subsection{ Ecuación diferencial de segundo orden}

\begin{equation}
m \frac{d^2x}{dt^2} + c \frac{dx}{dt} + kx = 0
\end{equation}

---

\subsection{ Sistema de ecuaciones}

\begin{equation}
\begin{cases}
2x + 3y = 5 \\
4x - y = 1
\end{cases}
\end{equation}

---

\subsection{ Matriz}

\begin{equation}
A =
\begin{pmatrix}
1 & 2 & 3 \\
4 & 5 & 6 \\
7 & 8 & 9
\end{pmatrix}
\end{equation}

---

\subsection{ Ecuación de Schrödinger}

\begin{equation}
i\hbar \frac{\partial}{\partial t} \Psi(x,t) =
\left( -\frac{\hbar^2}{2m} \nabla^2 + V(x) \right)\Psi(x,t)
\end{equation}

\end{document}
