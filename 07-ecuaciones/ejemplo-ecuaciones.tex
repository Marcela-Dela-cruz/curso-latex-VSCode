\documentclass[12pt]{article}

% Paquetes necesarios
\usepackage[spanish]{babel}
\usepackage[utf8]{inputenc}
\usepackage{amsmath}

\begin{document}

\section{Ecuaciones en LaTeX}

\subsection{Ecuación en línea}
La famosa relación entre energía y masa se expresa como $E = mc^2$,
la cual es una de las ecuaciones más conocidas de la física.

\subsection{Ecuación centrada sin numerar}

\[
E = mc^2
\]

\subsection{Ecuación numerada}

\begin{equation}
E = mc^2
\label{eq:energia}
\end{equation}

Como se muestra en la ecuación~\ref{eq:energia}, la energía es proporcional
a la masa y al cuadrado de la velocidad de la luz.

\subsection{Fracciones, raíces y potencias}

\begin{equation}
E = \frac{1}{2}mv^2
\label{eq:cinetica}
\end{equation}

La ecuación~\ref{eq:cinetica} representa la energía cinética de un cuerpo.

\subsection{Ecuaciones de varias líneas}

\begin{align}
F &= ma \label{eq:newton} \\
E &= mc^2
\end{align}

La ecuación~\ref{eq:newton} corresponde a la segunda ley de Newton.

\end{document}
